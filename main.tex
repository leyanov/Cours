\documentclass[french]{article}

\usepackage[utf8]{inputenc}
\usepackage{siunitx}
\usepackage{mhchem}

\title{Radioactivité}
\author{NEWERA TUTORING}

\begin{document}

\section{Obectifs}
\begin{itemize}
\item Mettre en évidence et interpréter l'émission spontanée des particules par des noyaux radioactifs;
\item Connaître les applications de la radioactivité ainsi que ses inconvénients.
\end{itemize}
\textbf{Important}: contrairement aux réactions chimiques au cours desquelles les atomes interagissent au niveau de leurs électrons périphériques, les réactions nucléaires font intervenir les noyaux des différents atomes
\section{Le noyau atomique}
\subsection{Structure de l'atome}
L'atome est constitué d’électrons et de nucléons (protons et neutrons). Les nucléons occupent le noyau de l'atome. L'atome est électriquement neutre : Il possède autant de protons que d’électrons.\\\\
\begin{tabular}{|c|c|c|c|c|}
    \hline
     Particule & Symbole & Charge électrique (C) & Masse en (Kg) & Masse en (u)\\ \hline
     Electron & $e^-$ & $q_e=-e=\num{-1,601e-10}$ & $M_e = \num{9,109e-31}$ & $M_e=\num{5,485e-4}$\\ \hline
     Proton & P & $q_e=e=\num{1,672e-27}$ & $M_P = \num{9,109e-31}$ & $M_e=1,00728$\\ \hline
     Neutron & N & $q_0 = 0$ & $M_N = \num{1,674e-27}$ & $M_N=1,00866$\\ \hline
    
\end{tabular}
\subsection{Définitions}
\begin{itemize}
\item On appelle \textbf{nombre de charge} noté \textbf{Z}, le nombre de proton contenu dans le noyau;
\item On appelle \textbf{nombre de masse} noté \textbf{A}, le nombre de nucléons contenu dans le noyau d’un atome. Ainsi $A = N + Z$;
\item On appelle \textbf{nucléide} l’ensemble d’atomes dont les noyaux ont le même nombre de proton et le même nombre de neutron. On représente un nucléide par le symbole : \ce{^{A}_{Z}X}\\\\
\textbf{Exemple}: \ce{^{4}_{2}He}, \ce{^{235}_{92}U}
\item On appelle \textbf{isotope} d’un élément les atomes de cet élément ayant le même nombre de charge \textbf{Z}, mais de nombre de masse \textbf{A} différents.\\\\
\textbf{Exemple}: \ce{^{1}_{1}H}, \ce{^{2}_{1}H}, \ce{^{3}_{1}H} sont des isotopes de  l’élément hydrogène.
\item On appelle isobares les noyaux d’éléments différents ayant le même nombre de masse \textbf{A}.\\\\
\textbf{Exemple}: \ce{^{40}_{18}Ar}, \ce{^{40}_{20}Ca}.
\item On appelle isotones, les noyaux d’éléments différents ayant même nombre de neutrons.\\\\
\textbf{Exemple}: \ce{^{15}_{7}N}, \ce{^{16}_{8}N}.
\item L’énergie nucléaire est l’énergie qui provient des réactions entre les noyaux d’atomes.
\end{itemize}

\textbf{L’unité de masse atomique}\\
En physique atomique, les masses sont souvent exprimées en \textbf{unité de masse atomique} noté \textbf{u} ou \textbf{u.m.a} : L’\textbf{u.m.a} est égale à un douzième de la masse d’un atome de l’isotope 12 de l’élément carbone.$$1u = \num{18e-3}$$

\end{document}
